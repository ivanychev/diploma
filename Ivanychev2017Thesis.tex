
\documentclass[a4paper, 10pt]{article}
\usepackage{Iv_xstyles}
\usepackage{Iv_commands}
\usepackage{algorithm}
\usepackage[noend]{algpseudocode}

\makeatletter
\def\BState{\State\hskip-\ALG@thistlm}
\makeatother


\tracingmacros = 0
% \addbibresource[datatype=bibtex]{papers.bib}
\usepackage{multicol}
\setlength{\columnsep}{0.5cm}

\setlength{\columnseprule}{.4pt}
\newcommand{\latexcolumnseprulecolor}{\color{gray}}


\newcommand{\ivtitle}[3]{
    \begin{center}
    \vspace{2.6cm}
    \textsc{\LARGE #1}

    \vspace{0.6cm}
    { #2}

\texttt{\small
    \href{ #3 }{#3}
}
    \vspace{0.6cm}
    \end{center}
        \noindent\makebox[\linewidth]{\rule{\textwidth}{0.4pt}}
}

\newcommand{\be}{\mathrm{Be}}
\newcommand{\bi}{\mathrm{Bi}}
\hypersetup{
    colorlinks=true,
    linkcolor=blue,
    filecolor=magenta,
    urlcolor=cyan,
}

\lstset{basicstyle=\footnotesize\ttfamily,breaklines=true}

\setcounter{tocdepth}{1}

\begin{document}

\ivtitle{Выполнимость гипотезы простоты выборки для комбинированных
признаковых описаний в задаче классификации временных рядов}{Сергей Иванычев.}{sergeyivanychev@gmail.com}

\tableofcontents

\section{Введение}

В работе рассматривается задача классификации временных рядов в задаче
классификации действий человека по временным рядам, порождаемым датчиками носимых устройств.
Классификация временных рядов является частным случаем классификации
объектов сложной структуры. Из-за того, что подобные задачи возникают во
многих областях, например, в обработке сигналов, биологии, финансах,
метеорологии, существует довольно много техник ее решения.

В нашей работе нас интересует решение задачи классификации временных
рядов путем построения промежуточного признакового пространства. Этот метод
применим не только к задаче классификации классификации данных с носимых устройств,
так как к объектам сложной структуры можно свести соответствующие ряды из других задач.
В общем случае подход с промежуточным признаковым пространством разделим на два этапа.

\begin{itemize}
    \item На первом этапе для сегментов временных рядов, которые выступают
    в роли объектов (которые, вообще говоря, могут
    быть различной длины и даже частоты дискретизации) вычисляются некоторые статистики
    или добываются некоторые экспертные оценки. В результате на каждый объект
    мы имеем некоторый набор показателей одной природы и из одного пространства.
    \item Над вторичным пространством этих показателей (то есть преобразованными
    объектами) работает некоторый алгоритм классификации (например ...), который
    обучается на "вторичной" выборке.
\end{itemize}

Эти этапы зависимы, так как классификатор, используемый во втором этапе может
потребовать от обучающей выборки выполнимость некоторых гипотез и, в частности,
гипотезы простоты выборки, что может быть обепечено
только корректным первым этапом. Выполнимость гипотезы простоты выборки,
находящейся в промежуточном пространстве необходима для корректной работы алгоритмов
классификации.

В нашей работе мы рассматриваем при каких условиях отображение объектов сложной структуры
порождает \textit{простую} выборку, то есть случайную, однородную и независимую,
а также предлагаем пути построения соответствующей выборки.

\section{Обзор литературы}

Work in progress.

\section{Постановка задачи классификации}

Рассмотрим некоторый временной ряд, то есть функцию определенную на множестве
временных меток.

$$
S: T \to \R \text{ где } T = \{t_0, t_0 + d, t_0 + 2d \ldots\}, |T| < \infty
$$

Зададим некоторую ширину сегмента $k \in \N$, тогда объектом $s_i$ мы
назовем набор

$$
s_i = (S(t), S(t - d), S(t - 2d), \ldots, S(t - (k - 1)d)) \in \mathfrak{S}
$$

Необходимо восстановить зависимость $y = f(s), f: \mathfrak{S} \to \{+1, -1\}$.
Для этого задана обучающая выборка

$$
\mathfrak{D} = \{ (s_i, y_i) \}_{i=1}^l, \;\;\; y_i \in \{+1, -1\}
$$

а также функция потерь

$$
L(f(s), y)
$$

Таким образом мы решаем задачу оптимизации

$$
\hat{y} = \arg\min_{y \in Y} \sum_{i = 1}^l L(f(s_i), y_i)
$$

\subsection{Комбинированное признаковое описание}

Пусть $\mathfrak{G}$ --- множество функций вида
$g: \mathfrak{S} \to \R^m$, где $m = m(g)$, то есть это множество отображений
пространства объектов сложной структуры в пространство действительных чисел
некоторой размерности (для каждой функции размерность может быть своя). В
$G$ могут лежать, например

\begin{itemize}
    \item Множество моделей локальной аппроксимации сигнала
    \item Множество статистик
    \item Множество экспертных оценок каждого из сложных объектов
\end{itemize}

Возьмем конечный поднабор этих функций $G = [g_1\ldots g_k] \subset \mathfrak{G}$.
Обозначим сумму размерностей образов функций из набора как

$$
n_G \triangleq \mathrm{dim}(\mathrm{Im}(g_1)) +
\mathrm{dim}(\mathrm{Im}(g_2)) + \ldots +
\mathrm{dim}(\mathrm{Im}(g_k))
$$

Тогда $G$ индуцирует отображение $G: \mathfrak{S} \to \Theta \subset \R^{n_G}$,
причем в векторах образа первые $\mathrm{dim}(\mathrm{Im}(g_1))$ компонент соответствуют
образу $g_1$, следующие $\mathrm{dim}(\mathrm{Im}(g_2))$ соответствуют $g_2$
и так далее. $\Theta$ называется \textit{признаковым пространством}
объектов сложной структуры $\mathfrak{S}$. Тогда, мы можем искать $f$ в семействе
суперпозиций $h(g(\cdot), \gamma)$, где

\begin{itemize}
    \item $g$ --- это признаковое отображение
    \item $h(\cdot, \gamma)$ --- параметрическое отображение $\Theta$ в $\{+1, -1\}$,
    которое фактически соответствует некоторому алгоритму машинного обучения,
    параметризованного вектором гиперпараметров $\gamma$.
\end{itemize}

Для отображения из параметрического пространства задана функция ошибки
$L(h(g(s_i), \gamma), y_i)$, тогда задача разбивается на два этапа:

\begin{itemize}
    \item Поиск и вычисление отображения $X = \{g(\{s_i\}_{i=1}^l), y_i)\}$.
    \item Определение оптимальных парметров $\gamma$ в задаче оптимизации

    $$
    \hat{\gamma} = \arg\min_{\gamma} \sum_{i=1}^{l}L(h(x_i, \gamma), y_i)
    $$
\end{itemize}

Основное допущение, принимаемое в данном методе является допущение о том, что
выборка в признаковом пространстве объектов является простой. В данной работе
мы рассматриваем, для каких признаковых пространств это допущение справедливо,
а также предлагаем способы построения таких выборок.



\end{document}
